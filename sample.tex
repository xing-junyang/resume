%%%%%%%%%%%%%%%%%
% This is an sample CV template created using altacv.cls
% (v1.7, 9 August 2023) written by LianTze Lim (liantze@gmail.com). Compiles with pdfLaTeX, XeLaTeX and LuaLaTeX.
%
%% It may be distributed and/or modified under the
%% conditions of the LaTeX Project Public License, either version 1.3
%% of this license or (at your option) any later version.
%% The latest version of this license is in
%%    http://www.latex-project.org/lppl.txt
%% and version 1.3 or later is part of all distributions of LaTeX
%% version 2003/12/01 or later.
%%%%%%%%%%%%%%%%

%% Use the "normalphoto" option if you want a normal photo instead of cropped to a circle
% \documentclass[10pt,a4paper,normalphoto]{altacv}

\documentclass[10pt,a4paper,ragged2e,withhyper]{altacv}
\usepackage{ctex}
%% AltaCV uses the fontawesome5 and packages.
%% See http://texdoc.net/pkg/fontawesome5 for full list of symbols.
% Change the page layout if you need to
\geometry{left=1.25cm,right=1.25cm,top=1.5cm,bottom=1.5cm,columnsep=1.2cm}

% The paracol package lets you typeset columns of text in parallel
\usepackage{paracol}

% Change the colours if you want to
\definecolor{SlateGrey}{HTML}{2E2E2E}
\definecolor{LightGrey}{HTML}{666666}
\definecolor{DarkPastelRed}{HTML}{450808}
\definecolor{PastelRed}{HTML}{8F0D0D}
\definecolor{GoldenEarth}{HTML}{E7D192}
\colorlet{name}{black}
\colorlet{tagline}{PastelRed}
\colorlet{heading}{DarkPastelRed}
\colorlet{headingrule}{GoldenEarth}
\colorlet{subheading}{PastelRed}
\colorlet{accent}{PastelRed}
\colorlet{emphasis}{SlateGrey}
\colorlet{body}{LightGrey}

% Change some fonts, if necessary
\renewcommand{\namefont}{\Huge\rmfamily\bfseries}
\renewcommand{\personalinfofont}{\footnotesize}
\renewcommand{\cvsectionfont}{\LARGE\rmfamily\bfseries}
\renewcommand{\cvsubsectionfont}{\large\bfseries}


% Change the bullets for itemize and rating marker
% for \cvskill if you want to
\renewcommand{\cvItemMarker}{{\small\textbullet}}
\renewcommand{\cvRatingMarker}{\faCircle}
% ...and the markers for the date/location for \cvevent
% \renewcommand{\cvDateMarker}{\faCalendar*[regular]}
% \renewcommand{\cvLocationMarker}{\faMapMarker*}


% If your CV/résumé is in a language other than English,
% then you probably want to change these so that when you
% copy-paste from the PDF or run pdftotext, the location
% and date marker icons for \cvevent will paste as correct
% translations. For example Spanish:
% \renewcommand{\locationname}{Ubicación}
% \renewcommand{\datename}{Fecha}


%% Use (and optionally edit if necessary) this .tex if you
%% want to use an author-year reference style like APA(6)
%% for your publication list
% % When using APA6 if you need more author names to be listed
% because you're e.g. the 12th author, add apamaxprtauth=12
\usepackage[backend=biber,style=apa6,sorting=ydnt]{biblatex}
\defbibheading{pubtype}{\cvsubsection{#1}}
\renewcommand{\bibsetup}{\vspace*{-\baselineskip}}
\AtEveryBibitem{%
  \makebox[\bibhang][l]{\itemmarker}%
  \iffieldundef{doi}{}{\clearfield{url}}%
}
\setlength{\bibitemsep}{0.25\baselineskip}
\setlength{\bibhang}{1.25em}


%% Use (and optionally edit if necessary) this .tex if you
%% want an originally numerical reference style like IEEE
%% for your publication list
\usepackage[backend=biber,style=ieee,sorting=ydnt,defernumbers=true]{biblatex}
%% For removing numbering entirely when using a numeric style
\setlength{\bibhang}{1.25em}
\DeclareFieldFormat{labelnumberwidth}{\makebox[\bibhang][l]{\itemmarker}}
\setlength{\biblabelsep}{0pt}
\defbibheading{pubtype}{\cvsubsection{#1}}
\renewcommand{\bibsetup}{\vspace*{-\baselineskip}}
\AtEveryBibitem{%
  \iffieldundef{doi}{}{\clearfield{url}}%
}


%% sample.bib contains your publications
\addbibresource{sample.bib}

\begin{document}
\name{邢骏洋}
\tagline{Junyang Xing}
%% You can add multiple photos on the left or right
\photoR{2.8cm}{Anya}
% \photoL{2.5cm}{Yacht_High,Suitcase_High}

\personalinfo{%
  % Not all of these are required!
  \email{xingjunyang@smail.nju.edu.cn}
  \phone{+86 130 5766 6023}
  \mailaddress{江苏省南京市鼓楼区汉口路22号}
  \location{中国·江苏·南京}
  \homepage{www.xjynotes.top}
  \github{xing-junyang}
  %% You can add your own arbitrary detail with
  %% \printinfo{symbol}{detail}[optional hyperlink prefix]
  % \printinfo{\faPaw}{Hey ho!}[https://example.com/]

  %% Or you can declare your own field with
  %% \NewInfoFiled{fieldname}{symbol}[optional hyperlink prefix] and use it:
  % \NewInfoField{gitlab}{\faGitlab}[https://gitlab.com/]
  % \gitlab{your_id}
  %%
  %% For services and platforms like Mastodon where there isn't a
  %% straightforward relation between the user ID/nickname and the hyperlink,
  %% you can use \printinfo directly e.g.
  % \printinfo{\faMastodon}{@username@instace}[https://instance.url/@username]
  %% But if you absolutely want to create new dedicated info fields for
  %% such platforms, then use \NewInfoField* with a star:
  % \NewInfoField*{mastodon}{\faMastodon}
  %% then you can use \mastodon, with TWO arguments where the 2nd argument is
  %% the full hyperlink.
  % \mastodon{@username@instance}{https://instance.url/@username}
}

\makecvheader
%% Depending on your tastes, you may want to make fonts of itemize environments slightly smaller
% \AtBeginEnvironment{itemize}{\small}

%% Set the left/right column width ratio to 6:4.
\columnratio{0.6}

% Start a 2-column paracol. Both the left and right columns will automatically
% break across pages if things get too long.
\begin{paracol}{2}
\cvsection{教育经历}

\cvevent{软件工程~\emph{本科生}~Software Engineering \emph{Undergraduate}}{南京大学~软件学院 \\Nanjing University Software Institute}{2022.9 ——~2026.6(预计毕业)}{南京市鼓楼区汉口路22号}
\begin{itemize}
\item 综合学分绩:4.53/5.0~(本专业排名前~10\%)
\item 在《\emph{软件工程与计算I}》、《\emph{C语言程序设计基础}》、《\emph{离散数学}》、《\emph{计算机组织结构}》、《\emph{互联网计算}》等专业课获得90分以上成绩。
\end{itemize}

\cvsection{参与项目}

\cvevent{网络报文可视化解析平台}{2024年南京大学大学生创新训练项目}{2024.1 ——~至今}{}

\begin{itemize}
\item 一个互联网协议栈的辅助教学平台
\item 可以实时捕获并解析数据包,并以可视化的方式向用户展示数据包的结构,便于用户学习协议栈的知识。同时提供网络诊断工具、故障场景模拟、带宽限制等功能,方便用户进阶学习。
\item 使用Vue3构建前端页面,后端采取Java+Spring框架,底层编写C++库来操纵TAP网卡设备。

\end{itemize}

\divider

\cvevent{Lambda 演算解释器}{}{2023.3 ——~2023.6\quad\quad\github{xing-junyang/lambdaIntepreter}}{}
\begin{itemize}
  \item 一个支持Lambda演算的解释器。
  \item 用户可以输入Lambda演算表达式,解释器会对表达式进行求值,并输出求值结果。用户也可以自定义函数,解释器会将函数定义存储在环境中,供后续调用。
  \item 使用Java进行开发。
\end{itemize}

\divider

\cvevent{恋之日计}{}{2024.1 ——~2024.2}{}
\begin{itemize}
  \item 一个情侣两人共享的日程管理小程序。支持课程表、日程表、倒数日等功能。
  \item 使用微信小程序框架构建。
\end{itemize}
\medskip

% use ONLY \newpage if you want to force a page break for
% ONLY the current column
\newpage

%% Switch to the right column. This will now automatically move to the second
%% page if the content is too long.
\switchcolumn

\cvsection{个人亮点}

\cvevent{}{获得奖项}{}{}
\begin{itemize}
\item 南京大学人民奖学金一等奖(2023.11)
\item 南京大学软件学院EL程序设计大赛算法组二等奖(2023.6)
\item “盛会启新程,泼墨展新篇”南京大学第七届微展示大赛优胜奖(2022.12)
\item 2020 CCF CSP-S 二等奖(2020.11)
\end{itemize}

\cvevent{}{荣誉称号}{}{}
\begin{itemize}
  \item 南京大学新生学院优秀学生(2023.6)
\end{itemize}

\medskip

\cvsection{座右铭}

\begin{quote}
``Talk is cheap. Show me the code.''\\
\quad\quad\quad\quad\quad\quad------ Linus Torvalds
\end{quote}

\cvsection{技术能力}
\cvtag{Java}
\cvtag{C/C++}
\cvtag{Assembly}
\cvtag{Python}
\cvtag{Vue3}
\cvtag{Go}
\cvtag{MySQL}

\divider\smallskip

\cvtag{Git}
\cvtag{Linux}
\cvtag{Docker}
\cvtag{Spring}\\
\cvtag{WeChat MiniProgram}


\cvsection{语言能力}

\cvskill{中文}{5}
\divider

\cvskill{英语/English}{3.5}
\divider

\cvskill{粤语/廣東話}{1} %% Supports X.5 values.

%% Yeah I didn't spend too much time making all the
%% spacing consistent... sorry. Use \smallskip, \medskip,
%% \bigskip, \vspace etc to make adjustments.
\medskip

% \cvsection{Most Proud of}

% \cvachievement{\faTrophy}{Fantastic Achievement}{and some details about it}

% \divider

% \cvachievement{\faHeartbeat}{Another achievement}{more details about it of course}

% \divider

% \cvachievement{\faHeartbeat}{Another achievement}{more details about it of course}

\end{paracol}


\end{document}
